%%%%%%%%%%%%%%%%%%%%%%%%%%%%%%%%%%%%%%%%%
% Wenneker Assignment
% LaTeX Template
% Version 2.0 (12/1/2019)
%
% This template originates from:
% http://www.LaTeXTemplates.com
%
% Authors:
% Vel (vel@LaTeXTemplates.com)
% Frits Wenneker
%
% License:
% CC BY-NC-SA 3.0 (http://creativecommons.org/licenses/by-nc-sa/3.0/)
% 
%%%%%%%%%%%%%%%%%%%%%%%%%%%%%%%%%%%%%%%%%

%----------------------------------------------------------------------------------------
%	PACKAGES AND OTHER DOCUMENT CONFIGURATIONS
%----------------------------------------------------------------------------------------

\documentclass[11pt]{scrartcl} % Font size

\input{structure.tex} % Include the file specifying the document structure and custom commands

%----------------------------------------------------------------------------------------
%	TITLE SECTION
%----------------------------------------------------------------------------------------

\title{	
	\normalfont\normalsize
	\textsc{CSCI 620: Introduction to Big Data}\\ % Your university, school and/or department name(s)
	\vspace{5pt} % Whitespace
	\rule{\linewidth}{0.5pt}\\ % Thin top horizontal rule
	\vspace{20pt} % Whitespace
	{\huge Activity 5 : Data Classification}\\ % The assignment title
	\vspace{12pt} % Whitespace
	\rule{\linewidth}{2pt}\\ % Thick bottom horizontal rule
	\vspace{12pt} % Whitespace
}

\author{\LARGE Ameya Nagnur (an4920)\\ \LARGE Siddarth Sargunaraj (sxs2469)\\ \LARGE Ketan Kokane (kk7471)\\ \LARGE Kavya Kotian (kk2014)} % Your name


\date{\normalsize\today} % Today's date (\today) or a custom date

\begin{document}

\maketitle % Print the title

%----------------------------------------------------------------------------------------
%	FIGURE EXAMPLE
%----------------------------------------------------------------------------------------
\section{}
% \begin{left}
    
% \begin{tabular}{ | c | c  c  c | c | }
% \hline
% \textbf{Instance} & \textbf{$a_1$} & \textbf{$a_2$} & \textbf{$a_3$} & \textbf{Target Class} \\
% \hline
% 1.0 & a & s & s & g\\
% \hline
% 3.0 & a & s & s & g\\
% \hline
% 4.0 & a & s & s & g\\
% \hline
% 5.0 & a & s & s & g\\
% 5.0 & a & s & s & g\\
% \hline
% 6.0 & a & s & s & g\\
% \hline
% 7.0 & a & s & s & g\\
% 7.0 & a & s & s & g\\
% \hline
% 8.0 & a & s & s & g\\
% \hline

\end{tabular}
\end{left}
\subsection{What is the entropy of this collection of training examples with respect to the positive class}

\subsection{What are the information gains of splitting on a1 and splitting on a2 relative to these training examples?}
\subsection{For a3, which is a continuous attribute, compute the information gainfor every possible split}
\begin{left}
\begin{tabular}{ | c | c | c | c | c | }
\hline
\textbf{$a_3$} & \textbf{Split Point} & \textbf{Class Labels} & \textbf{Entropy} & \textbf{IG} \\
\hline
1.0 & a & s & s & g\\
\hline
3.0 & a & s & s & g\\
\hline
4.0 & a & s & s & g\\
\hline
5.0 & a & s & s & g\\
5.0 & a & s & s & g\\
\hline
6.0 & a & s & s & g\\
\hline
7.0 & a & s & s & g\\
7.0 & a & s & s & g\\
\hline
8.0 & a & s & s & g\\
\hline


\end{tabular}
\end{left}
%-------------------------------------------------------------------
\subsection{What is the best split (among a1, a2, and a3) according to the information gain?}
\subsection{What is the best split (between a1 and a2) according to the classification error rate}

%------------------------------------------------
\end{document}
